%%%%%%%%%%%%%%%%%%%%%%%%%%%%%%%%%%%%%%%%%%%%%%%%%%%%%%%%%%%%%

%------------------------------------------------------------
%
\documentclass[a4paper, notitlepage]{article}%
%Options -- Point size:  10pt (default), 11pt, 12pt
%        -- Paper size:  letterpaper (default), a4paper, a5paper, b5paper
%                        legalpaper, executivepaper
%        -- Orientation  (portrait is the default)
%                        landscape
%        -- Print size:  oneside (default), twoside
%        -- Quality      final(default), draft
%        -- Title page   notitlepage, titlepage(default)
%        -- Columns      onecolumn(default), twocolumn
%        -- Equation numbering (equation numbers on the right is the default)
%                        leqno
%        -- Displayed equations (centered is the default)
%                        fleqn (equations start at the same distance from the right side)
%        -- Open bibliography style (closed is the default)
%                        openbib
% For instance the command
%           \documentclass[a4paper,12pt,leqno]{article}
% ensures that the paper size is a4, the fonts are typeset at the size 12p
% and the equation numbers are on the left side
%-------------------------------------------



\begin{document}

\title{People are sensitive to hypothesis sparsity during category discrimination}
\author{{\bf Steven Langsford*} \\
   	{\bf Drew Hendrickson} \\
	{\bf Amy Perfors} \\
   	{\bf Daniel J. Navarro}\\
   School of Psychology, University of Adelaide\\~\\Contact: langsford.steven@gmail.com}
\date{}

\maketitle
\thispagestyle{empty}
\pagestyle{empty}

\begin{abstract}
People's sensitivity to expected information value when choosing between two different types of information request was examined in a simple category learning task. Previous work has shown that the information value of requests can be manipulated by controlling the \textit{sparsity} of hypotheses, the degree to which category members are rare or common in the domain under consideration when making those requests. However the degree to which people are sensitive to expected information value is unknown. This study examined a binary sorting task where sparsity differed across conditions. In contrast to previous work using visual areas to represent the coverage of a domain by a hypothesis, the stimuli in this study defined hypotheses in an abstract similarity space over geometric shapes. Participants were able to request labels for either category members or non-members. While both request types were used in all conditions, most often evenly, the proportion of participants showing a preference for one type of request was strongly impacted by the information value of that request type. A small tendency to prefer requests from the designated target category was also observed.
\end{abstract}

\end{document}
