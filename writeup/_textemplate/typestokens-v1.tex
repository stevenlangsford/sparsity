% 
% Annual Cognitive Science Conference
% Sample LaTeX Paper -- Proceedings Format

%% Change ``a4paper'' in the following line to ``letterpaper'' if you are
%% producing a letter-format document.

\documentclass[10pt,letterpaper]{article}

\usepackage{cogsci}
\usepackage{pslatex}
\usepackage{apacite}
\usepackage{graphicx}
\usepackage{amsmath,amsfonts}
\usepackage{tipa}

\usepackage{caption}[2005/10/24]
\newcommand{\smallspace}{\def\baselinestretch{1.1}}
\DeclareCaptionFont{smallspace}{\smallspace}
\captionsetup{
   margin   = {0pt},
   font        =  {footnotesize,smallspace},
   aboveskip = {3pt},
   belowskip = {-10pt},
   labelfont = {up},
   textfont  = {up},
}

\title{Sensitivity to types vs. tokens in linguistic and non-linguistic generalization}
 
\author{{\bf Amy Perfors} (amy.perfors@adelaide.edu.au) \\
   {\bf Keith Ransom} (keith.ransom@adelaide.edu.au) \\
   {\bf Daniel J. Navarro} (daniel.navarro@adelaide.edu.au) \\
   School of Psychology, University of Adelaide}


\begin{document}

\maketitle


\begin{abstract}
Insert abstract here.\\
\textbf{Keywords:} 
generalization; grammar learning; adaptor grammar; types; tokens; frequency; size principle
\end{abstract}


\section*{Introduction}

The problem of induction, or how people generalize from limited data, is a central one in cognitive science. A core aspect of the problem is what people think about data they have not seen. For instance, if someone has seen three examples of the concept {\sc dog} (two dalmations and a terrier), they must decide if another, different item -- say a spaniel -- is also a dog.  Similarly, people learning a language have heard certain sentences and must then decide which sentences that they haven't heard are also permitted by the grammar. The {\it tightness} of a generalization reflects one's willingness to extend the concept to cover increasingly different exemplars: a person willing to call a siamese cat a dog is generalizing somewhat loosely, while a person who only accepts dalmations and a terrier as dogs is generalizing very tightly. Neither person's inferences contradict any of the observed data, which implies that how tight generalizations are is guided by something other than logical necessity. 

The mathematics of probability theory provide one normative standard for how generalizations should tighten with additional data. As long as one believes that exemplars are strongly sampled (i.e., sampled from the concept, as opposed to sampled from the world and then labelled), then the probability of any one data point is $\frac{1}{n}$, where $n$ is the number of items in the concept. In other words, if there are only three possible {\sc dogs} in the world (dalmatians, terriers, and spaniel) then the probability of seeing any one of them at any one time is $\frac{1}{3}$. This principle is known as the {\bf size principle}, and it -- or something like it -- has been shown to guide human generalization in a variety of contexts (CITE).

However, all of the research so far concentrates on situations in which each of the data points is distinctly different from previous ones -- as if one had sampled three dogs and had been given a dalmation, a spaniel, and a terrier rather than two dalmatians and a terrier. What should people do when each data point looks superficially identical? This happens all of the time in real life: children view the same dog multiple times, language learners hear multiple utterances of the same sentence (``Hello! How are you!"), and so forth. When each data point looks superficially identical, what is the proper normative thing to do? This depends on how one interprets the generative process behind the data. If sampling is assumed to be roughly similar to drawing examples with replacement from a bag of possibilities, then the size principle should apply: seeing two dalmatians and a terrier should result in greater tightening than seeing one dalmatian and a terrier: the probability of the data in the first case would be $\frac{1}{n^3}$ while the probability in the second case is only $\frac{1}{n^2}$.

Another possibility, however, is that people might treat identical instances (tokens) as informative about the frequency distribution that they should expect to see, but only the number of different distinct types as informative about the extent to which they should generalize. This sort of inference is sensible as long as one views identical exemplars as merely multiple views of the same (single) data point. In the linguistic domain, it is captured formally in the {\it adaptor grammar} framework (CITE), which suggests that the learner may make a distinction between the element that licenses which items are allowed (the grammar) and the element (somewhat like a memory cache) that affects the frequency of those items (the adaptor). This framework has been successfully applied to many aspects of language learning, from morphology (CITE) to syntax (CITE). If people assume that the grammar and the adaptor are entirely separate, this implies that seeing additional identical tokens should not lead to tighter generalizations -- it will affect what people expect in terms of the frequency distribution of items, but inferences about the grammar itself would only be affected by additional types. This analysis thus predicts that receiving additional identical data points should not result in tighter generalizations at all.

While the adaptor grammar framework was designed and has only been implemented in the domain of language, there is no reason that the general logic should not apply more widely. Concepts, like languages, can be decomposed into an underlying core that licenses which items are allowed in the concept (we might call this an {\it extension} instead of a {\it grammar}) which is logically separable from an additional adaptor that imposes a frequency distribution on the allowed items. That said, language is different from concept learning in many ways -- most relevantly for this analysis, individual sentences are not physical entities in the same way that individual exemplars from a concepts are. Thus it may be far more natural for people to ignore the token frequency of particular sentences but not ignore the token frequency of particular exemplars of the concept {\sc dog}. 

This paper investigates whether people tighten their generalizations upon seeing additional identical exemplars of previously-seen data, and whether they behave differentially if the domain is linguistic or non-linguistic. We begin by presenting an experiment in which participants were shown a dataset of 10 distinct types of exemplars, each occurring either once or ten times. Our main question is whether people tighten their generalizations when they are shown ten times as much data, even though the number of types is equivalent in each case. We also investigate whether this tendency varies by domain by presenting the same situation within a linguistic and a non-linguistic (category-learning) surface form. We find that in both domains, there is no difference in generalization with increasing data. Furthermore, although generalizations tighten somewhat when people are given assistance remembering all of the items they see, a formal analysis within the adaptor grammar framework suggests that the vast majority of participants are better fit by assuming that they generalize based only on types, rather than also on token frequency. We conclude with a discussion of implications for linguistics and concept learning.


\section*{Experiment}

%{\bf Participants}. 
454 adults were recruited via Amazon Mechanical Turk. %, an online resource increasingly used and validated for experiments in psychology and linguistics \cite{sprouse11,crumpetal13}. 
%The experiment varied four factors, described in more detail below: the {\sc type} of the stimuli, the {\sc quantity} of training data, the extent to which participants had assistance with their {\sc memory}; and the {\sc salience} of the stimuli. Each factor varied in two ways, resulting in a 2x2x2x2 design.  
40 participants were excluded from further analysis for failing to pass a ``check'' question, described below.  This resulted in 414 total participants, split approximately equally between all conditions. % (lowest: N; highest N). Between 189 and 210 participants corresponded to each of the possible assignments to the four factors. 
Ages ranged from 18 to 66 (mean: 31.8) and 39.4\% were female. 314 of the final participants were from the United States and 68 were from India. Those remaining were from 12 other countries in Africa, North and South America, Europe, and Asia. All participants were paid \$0.50US for the 5-10 minute experiment.

%\begin{figure*}[t]
%\begin{center}
%\includegraphics[scale=0.43]{figs/allstimuli.pdf}
%\end{center}
%\caption{Training and test stimuli in the {\sc inscription} and {\sc design} conditions. Participants in the {\sc design} condition were told that they were to classify bracelets with different patterns on them, while those in the {\sc inscription} condition were told that they were learning bracelets with different inscriptions. The goal was for those in the {\sc inscription} condition to treat the stimuli as linguistic, while those in the {\sc design} condition would treat them more as objects to be classified. {\bf Left table}. This shows each of the 10 training stimulus types. Stimuli were generated from a grammar of the form $A^nB^mA^n$, where $A = \{du\}$ and $B=\{bo,gi,la\}$. These items occurred once each in the {\sc 1x} condition and ten times each in the {\sc 10x} condition. {\bf Right table}. This shows the 15 test stimuli, listed in decreasing order according to how closely they match the training data. The top stimuli ({\sc observed}) precisely match stimuli that were seen in the input. The {\sc depth-limited} stimuli could have been generated by a grammar approximating the $A^nB^mA^n$ grammar, limited to the depth of embedding as the training stimuli. The {\sc CFG} sentences could be generated by that grammar without that limitation on depth of embedding. The {\sc any-order} stimuli could be generated by a grammar that allows $A$ or $B$ elements in any order; this grammar could have generated the training stimuli but also many other sentences as well. Finally, as a control, the {\sc incorrect} stimuli could have been generated by a grammar with a different underlying vocabulary.}
%\label{stimuli}
%\end{figure*}

\begin{figure}[t]
\begin{center}
\includegraphics[scale=0.23]{figs/stimuli.pdf}
\end{center}
\caption{Sample stimuli in the {\sc inscription} and {\sc design} conditions. Participants in the {\sc design} condition were told that they were to classify bracelets with different patterns on them, while those in the {\sc inscription} condition were told that they were learning bracelets with different inscriptions. The goal was for those in the {\sc inscription} condition to treat the stimuli as linguistic, while those in the {\sc design} condition would treat them more as objects to be classified. The top two rows show stimuli from training; the bottom three show stimuli from testing.}
\label{stimuli}
\end{figure}

\subsection{Procedure}

After completing basic demographic information, participants were told that in this experiment they were acting as curators of a museum who have some bracelets in their collection which they received from their predecessor, and which all come from the same place. All participants then answered a series of multiple-choice questions to make sure they had read and understood the instructions.  

The experiment had two phases. The first was a training phase in which people were shown sample bracelets from their collection one-by-one, clicking \textsf{Next} to see the next item. The appearance and number of the bracelets, as well as whether previously-viewed ones stayed on screen, varied by condition. In the second phase, people were shown new bracelets and asked to rate on a scale of 1 to 7 whether they think the new bracelet belongs in this collection (1 = ``Agree strongly'' and 7 = ``Disagree strongly''). There were 15 test items which varied according to how closely they matched the original stimuli (described in more detail below).

\subsection{Conditions}

This experiment varied three factors\footnote{We varied a fourth factor, saliency, by varying whether they were colored or not. Because this manipulation did not produce interesting effects, for space reasons we do not report on it.}, resulting in a 2x2x2 design and 8 conditions. We describe each factor below.

{\sc Type}. One of interest was whether people generalize differently depending on whether they see the stimuli as linguistic or not. In the {\sc inscription} condition, participants were told that the bracelets each contained an inscription that they would read. In the {\sc design} condition, participants saw the jewelled pattern of the bracelet, which was designed to be of similar size and complexity as the inscription (see Figure~\ref{stimuli}). In order to make the conditions comparable the patterns in the {\sc design} condition directly corresponded to the inscriptions in the {\sc inscription} condition. However, we hoped that the cover story and bracelet-like appearance of the stimuli would make this as little like a linguistic task as possible. 


\begin{table}[t]
\begin{center}
\begin{tabular}{|c|}
\hline
du gi bo du \\
du la la gi du\\
du gi gi bo la du \\
du gi la gi bo du \\
du du bo du du \\
du du gi bo gi du du \\
du du la bo gi gi du du \\
du du du gi la du du du \\
du du du bo gi la du du du \\
du du du du bo du du du du \\
\hline
\end{tabular}
\caption{Each of the 10 training stimulus types in the {\sc inscription} condition. Stimuli were generated from a grammar of the form $A^nB^mA^n$, where $A = \{du\}$ and $B=\{bo,gi,la\}$. Stimuli in the {\sc design} condition corresponded exactly to these; examples are shown in Figure~\ref{stimuli}. These items occurred once each in the {\sc 1x} condition and ten times each in the {\sc 10x} condition.}
\label{stimuluslist}
\end{center}
\end{table}


{\sc Quantity}. The major question motivating this work was whether people tighten their generalizations with additional instances of identical exemplars. We therefore varied the quantity of training stimuli people received. In the {\sc 1x} condition, people saw 10 distinct stimulus {\it types}, shown in Table~\ref{stimuluslist}. Each type was generated from the grammar $A^nB^mA^n$, where $A = \{du\}$ and $B=\{bo,gi,la\}$.\footnote{In the {\sc design} condition people saw patterns that exactly corresponded to these syllables, as in Figure~\ref{stimuli}. For ease of reference henceforth we will just refer to them as they appear in the {\sc inscription} condition.} This grammar is a context-free grammar (CFG), but as we will see in more detail, the specific 10 stimulus types can be captured with varying degrees of exactness by grammars of various sorts; which grammar is inferred will affect which additional stimuli are accepted as members of the same category. As is typical with a CFG, there are fewer stimuli at higher depths of embedding. 

The {\sc 10x} condition was identical to the {\sc 1x} condition except that people saw 10 exemplars of each of the 10 types (100 stimuli in total). If people are paying attention only to the distinct types when forming generalizations, we would expect performance to be identical in the {\sc 1x} and {\sc 10x} conditions, despite the fact that there is ten times more data in the latter. On the other hand, if people form generalizations on the basis of token frequency as well, we would expect them to generalize far less -- to accept many fewer test stimuli as acceptable category members -- in the {\sc 10x} condition. The order of all stimuli was completely random.

\begin{table}[t]
\begin{center}
\begin{tabular}{|c|c|}
\hline
Stimulus & Type \\
\hline
du la la gi du & Observed \\
du du la bo gi gi du du & Observed \\
du du du gi la du du du & Observed \\
du bo gi la la du & Depth-limited \\
du du du la du du du & Depth-limited \\
du du la gi bo du du & Depth-limited \\
du du du du du du bo la du du du du du du & Full CFG \\
du du du du du gi bo du du du du du & Full CFG \\
du du du du du la du du du du du & Full CFG \\
bo du gi gi la bo & Any order \\
du du du la bo du & Any order \\
gi du du la du la & Any order \\
wi sa fo & Incorrect \\
fo wi pe wi wi ho vu & Incorrect \\
pe ho sa vu vu re & Incorrect\\
\hline
\end{tabular}
\caption{Test stimuli, listed in decreasing order according to how closely the match the training data. The top stimuli ({\sc observed}) precisely match stimuli that were seen in the input. The {\sc depth-limited} stimuli could have been generated by the $A^nB^mA^n$ grammar, limited to the depth of embedding as the training stimuli. The {\sc full CFG} sentences could be generated by that grammar without that limitation. The {\sc any order} stimuli could be generated by a grammar that allows $A$ or $B$ elements in any order; this grammar could have generated the training stimuli but also many other sentences as well. Finally, the {\sc incorrect} stimuli could have been generated by a grammar with a different underlying vocabulary.}
\label{teststimuli}
\end{center}
\end{table}


{\sc Memory aid}. Because the extent to which one generalizes is in part a function of one's memory for the training data, we varied the degree to which people had to rely on their memory to do this task. In the {\sc memory-aided} condition, each of the training stimuli remained onscreen after the participants clicked \textsf{Next}; previous stimuli were shown smaller (but still legibly) in the background. They remained onscreen while the test questions were being answered as well. The {\sc memory-unaided} condition was more like a typical category-learning experiment: people saw each stimulus one-by-one, and it disappeared before the next stimulus appeared. Of interest is whether people generalize less in the {\sc memory-aided} condition, in particular if there is an interaction with the {\sc quantity} or {\sc type} manipulations.

%{\sc Salience}. Because our underlying grammar hinges critically on the number of $du$ syllables on either end, we thought the salience of these outer syllables might affect people's generalizations. Therefore, in the {\sc color} condition the outer $du$ (or corresponding pattern) was blue while the rest of the stimulus was black; conversely, in the {\sc monocolor} condition the stimuli were entirely black. This manipulation was less interesting on its own and did not turn out to have any interactions with the others, so for the sake of brevity all subsequent analyses are collapsing this data together.

%One of the goals with this experiment is to estimate which of the grammars under consideration are closest DO I WANT TO FIT INDIVIDUAL PARTICIPANTS TO GRAMMARS? MIGHT BE KIND OF COOL. 


\subsection{Test stimuli}

An essential part of this research is to be able to evaluate how tightly or loosely people generalize from the training stimuli they have seen. To that end, we constructed test stimuli that could have been generated by grammars that more or less precisely fit the input data. All stimuli are shown in Table~\ref{teststimuli}, and are described in detail in this section.

\textsf{Observed}. These stimuli occurred in the training data. They therefore represent the tightest generalization, and we expected that participants should consistently accept them. % as possible bracelets.

\textsf{Depth-limited}. These could have been generated by a grammar approximating the $A^nB^mA^n$ grammar, but limited to the same depth of embedding as the training stimuli.\footnote{Because of the limitation in depth, this grammar might therefore be implementable as a regular grammar.}  It represents a tight level of generalization: people endorsing these stimuli but not {\sc full CFG} would have realized that the number of elements on the left and right must match, but would not think that there could be more than four elements (since there were never more than four during training).

\textsf{Full CFG}. These stimuli could have been generated by the $A^nB^mA^n$ grammar without that limitation on depth of embedding; the left and right elements occur more often than was observed during training. As such, accepting these stimuli requires generalizing further away from the training data.

\textsf{Any order}. These stimuli could be generated by a grammar with the same underlying $A$ or $B$ elements, but permits them to occur in any order. Because it captures the training stimuli, it is not wrong, but accepting these stimuli amounts to generalizing quite far from the training.

\textsf{Incorrect}. These stimuli could be generated by a grammar with a different underlying ``vocabulary'' (i.e., different syllables or bracelet patterns). Accepting them requires generalizing very far from the training data. We therefore used these stimuli as a ``check'' to catch those participants who were not trying or did not understand the task. The 40 participants excluded from the analysis were those who agreed that these stimuli belonged in the collection (giving them a rating of 1, 2, or 3 on the 7-point scale described earlier). 


\section{Results}

\begin{figure*}[t]
\begin{center}
\begin{tabular}{ccc}
\includegraphics[scale=0.2]{data/typegen.pdf} & \hspace{-5mm}\includegraphics[scale=0.2]{data/quantitygen.pdf} &  \hspace{-5mm}\includegraphics[scale=0.2]{data/memorygen.pdf} \\ 
\end{tabular}
\caption{Results.}
\label{rawgeneralization}
\end{center}
\end{figure*}

Figure~\ref{rawgeneralization} shows the average degree of generalization by each of the three main factors. Two-way ANOVAs showed that for all three factors, there was a significant main effect of test stimulus ({\sc type}: $F(4,6200)=1662.6, p<0.0001$; {\sc quantity}: $F(4,6200)=1662.6, p<0.0001$; {\sc memory}: $F(4,6200)=1678.3, p<0.0001$). People responded significantly differently to the different test stimuli, generalizing more to the ones that are more similar to the training stimuli and less to the ones that are different. This is as we expected and is a clear indication that people understood the task.

More relevantly to the main questions motivating this work, there is no main effect of the type or quantity of stimulus ({\sc type}: $F(1,6200)=0.01, p=0.913$; {\sc quantity}: $F(1,6200)=0.07, p=0.786$). Overall, people generalized the same regardless of whether they were classifying bracelets according to the {\sc inscription} or the {\sc design}, and regardless of whether they saw 10 or 100 data points. That said, the interaction for both factors was significant ({\sc type}: $F(4,6200)=4.14, p=0.002$; {\sc quantity}: $F(4,6200)=4.13, p=0.002$). Based on Figure ~\ref{rawgeneralization}, it appears that people are slightly more willing to accept the \textsf{Depth-limited} stimuli in the {\sc inscription} and {\sc 10x} condition, but slightly less willing to generalize more broadly in those conditions. The effect size is tiny, suggesting that these differences are small.

Receiving a memory aid, however, makes a larger difference: there is a significant main effect of having a memory aid ($F(1,6200)=22.52, p<0.0001$) and a significant interaction ($F(4,6200)=13.12, p<0.0001$). Not surprisingly, people were more likely to accept the \textsf{Observed} sentences if they could see the identical training stimuli on their screen, thanks to the memory aid. They were also less likely to accept the other test sentences. This is also not surprising, since people who must rely on their memory may not recall clearly if they have seen stimuli like the test stimuli in the training, and thus may be more willing to accept them.

Does memory mediate the effect of stimulus quantity? One might expect that people would be more affected by a greater quantity of data in the {\sc memory-aided} condition, because they could remember all of the extra data better. Within the {\sc memory-aided} condition, there is a main effect of test stimulus ($F(4,2825)=766.0, p<0.0001$), no effect of quantity of data ($F(1,2825)=1.1, p=0.286$), and a small interaction ($F(4,2825)=3.1, p=0.014$). Within the {\sc memory-unaided} condition, the main effects are the same (test stimulus: $F(4,3365)=932.6, p<0.0001$; quantity: $F(1,3365)=0.8, p=0.369$) but the interaction is stronger ($F(4,3365)=4.9, p=0.0006$). This suggests that although the basic effect of stimulus quantity holds regardless of whether there is a memory aid, it is exacerbated when there is not. 

So far these findings suggest that people generalize differently when given additional identical data or when the stimuli look different.

%This is in keeping with the interpretation of the ``adaptor'' part of the adaptor grammar reflecting memory: when people are forced to rely on their memory (wait, is it?)



Now motivate the modelling analysis, as a desire to see if more people are more type-based or token-based? Look at other interactions? (Will take up a lot of space).

\section{Discussion}

Was the bracelet in the {\it design} condition really non-linguistic, or did people interpret it as a script in an unknown language?

Something about the adaptor explaining the memory aid stuff

Barsalou et al appeared to find that generalisations about categories were on the basis of tokens. they were asked about specific items (one whose features were on the higher-frequency individual token, one whose features matched more of the types). the questions were which one is more likely to belong in the category and which one is more typical. 
     in general it feels like they are doing something different here - what? aren't looking at tightness of generalisations. tightness of generalisation is about in general how similar the items have to be before they are okay - what we show is that token frequency doesn't affect it that much. they are looking at whether specific items, which match more on tokens or types, are likely to be included. if people are interpreting that question to be about the adaptor part then it is totally possible to observe that.
    also, it's a very different kind of category (and possibly not entirely category-like).

Also should draw the link to my stuff on hierarchical phrase structure. Point out it only makes sense, because generalising by tokens implies that as people get older they should be increasingly unwilling to generalize at all beyond what they have heard (token frequency gets immense). But we know people of all ages are willing to generalize new constructions.

\small
\section{Acknowledgments}

Thank you to Simon De Deyne and Natalie May for their help in designing and running pilot versions of the experiment. This research was supported by ARC grants DE120102378 and DP110104949. 

\renewcommand{\bibliographytypesize}{\footnotesize}
\bibliographystyle{apacite}

\setlength{\bibleftmargin}{.125in}
\setlength{\bibindent}{-\bibleftmargin}


\bibliography{amybib}


\end{document}

